\RequirePackage{filecontents}
\begin{filecontents}{references.bib}
@techreport{bouma2007corea,
  title={The COREA-project, manual for the annotation of coreference in Dutch texts},
  author={Bouma, Gosse and Daelemans, Walter and Hendrickx, Iris and Hoste, V{\'e}ronique and Mineur, A},
  year={2007},
  institution={University of Groningen},
  note={\url{https://www.clips.uantwerpen.be/~iris/corea/publications/manual_corea.aug07.pdf}}
}

@techreport{schoen2014newsreader,
  title={NewsReader Document-Level Annotation Guidelines-Dutch},
  author={Schoen, Anneleen and van Son, Chantal and van Erp, Marieke and van Vliet, Hennie},
  year={2014},
  institution={VU University},
  note={\url{http://www.newsreader-project.eu/files/2013/01/8-AnnotationGuidelinesDutch.pdf}}
} 
\end{filecontents}
\PassOptionsToPackage{hyphens}{url}
\documentclass[a4paper]{article}
\usepackage[T1]{fontenc}
\usepackage[utf8]{inputenc}
\usepackage{kpfonts, mdwlist, microtype, xcolor, natbib}
\usepackage[unicode=true]{hyperref}
\hypersetup{pdfborder={0 0 0}, breaklinks=true}
\setlength{\emergencystretch}{3em}  % prevent overfull lines
\newcommand{\n}[1]{\textcolor{red}{#1}}

\title{Annotation guidelines for dutchcoref}
\author{Andreas van Cranenburgh}
\date{}

\begin{document}
\maketitle

\section{How to annotate?}

\begin{itemize*}
\item Read the text from start to finish, make and correct annotations as
  you go.
\item Identify mentions by asking yourself whether a span of text describes
  a specific identifiable object or person.
\item When the same entity is referred to again, ensure that both mentions
  are in the same coreference cluster. Conversely, remove any
  incorrect links.
\end{itemize*}

\section{Mentions}

A mention is a span of text that refers to an entity or person in the
real or mental world. All mentions referring to objects or persons are
annotated, including entities that are not referred to again
(singletons). Mentions have been automatically identified, but they may
need to be corrected.

The following subsections list the types of mentions that should be annotated.
The span of a mention is indicated here with square brackets {[} and {]};
a span that should not be annotated as mention is indicated with \n{[} red brackets\n{]}.

\subsection{Pronouns}
\begin{itemize*}
    \item Personal pronouns (\emph{zij, hun}, \dots).
        Includes \emph{het} when used as pronoun.
    \item Possessive pronouns (\emph{mijn, zijn}, \dots)
    \item Demonstrative pronouns (\emph{die, dat, deze, dit, daar})
    \item Relative pronouns (\emph{die, dat, wie, wat})
    \item Reflexive/reciprocal pronouns (\emph{zich, zichzelf, elkaar}).
        Both obligatory and normal reflexives are annotated.
    \item Indefinite/generic pronouns (\emph{men, je, ze, iedereen, iemand, \dots})
      when the same unspecified person/thing can be referred to again. This
      excludes e.g. \emph{niemand} or wh-pronouns in questions
      (\emph{wie, wat, welke, \dots)}.
    \item Pronominal adverbs of location: \emph{er, hier, daar, waar, waarin, \dots}
\end{itemize*}

Exclude non-referential, pleonastic pronouns:
\begin{itemize*}
   \item \emph{\n{[}Het\n{]} regent.}
   \item \emph{Daar moeten we \n{[}het\n{]} over hebben.}
   \item \emph{\n{[}Er\n{]} zit niets anders op.}
\end{itemize*}


\subsection{Proper nouns (named entities)}
\begin{itemize*}
    \item One-word names: \emph{[Jan], [Amerika]}.
    \item Multiword names form a single mention: \emph{[Jan de Vries], [de Verenigde Staten]}.
\end{itemize*}


\subsection{Noun phrases (NPs)}
Always annotate the longest, most specific
continuous span describing a mention. What to include:
\begin{itemize*}
    \item Determiners: \emph{[het huis]} \\
        A possessive pronoun is a determiner,
        and is also its own mention:\\
            \emph{[[mijn] fiets]}
    \item Adjectives, nouns: \emph{[een warme kop thee]}
    \item Prepositional phrases modifying the noun:
        \emph{[kandidaat voor [de coalitie]]}.
    \item Noun phrases within noun phrases. See previous example.
        Since \emph{kandidaat} and \emph{coalitie} describe different
        entities, they are both annotated. On the other hand,
        there is no need to mark \emph{kandidaat} twice:

        \emph{[\n{[}kandidaat\n{]} voor [de coalitie]]}.

\end{itemize*}

Special cases:
\begin{itemize*}
\item Conjunctions (\emph{Jan en Marie}). Include the whole conjunction as
  mention only when it functions as a unit in the text;
  e.g., when referred to again as a single group bij a plural pronoun ``\emph{ze}''.
  By default, only the individual conjuncts \emph{Jan} and \emph{Marie} are
  considered as separate mentions.
\item NPs with commas.
    Except in special cases, a comma indicates the end of a mention:

    \emph{[De nieuwste iPhone]}, \emph{[een revolutionaire nieuwe smartphone]}.

    Special cases:
    \begin{itemize*}
        \item Geographical: \emph{{[}Los Angeles, California{]}}
        \item Adjective: \emph{{[}Een mooie, rode roos{]}}
        \item Conjunction functions as group (see above)
        
          \emph{{[}{[}Jan{]}, {[}Marie{]} en {[}Joost{]}{]}}
          
    \end{itemize*}

\item Discontinuous NPs

  \emph{\n{[}[een belediging] /zijn/ van onze gastvrijheid\n{]}}
  
    Mentions must be continuous, uninterrupted spans in the text.
    Since the verb ``\emph{zijn}'' is not part of the noun phrase,
    it should also not be part of the mention.
    In this case only ``\emph{een belediging}'' is marked as a mention
    (i.e., the part with the head of the constituent \emph{belediging}).

\item Relative clauses.
    The relative pronoun indicates the end of the mention:
  
    \emph{\n{[}[De burgemeester]\textsubscript{1} {[}die{]}\textsubscript{1} de vergadering opende\n{]} was behoorlijk nors.}

\end{itemize*}

What to exclude:

\begin{itemize*}
\item Time-related NPs: \emph{\n{[}gisteren\n{]} , \n{[}de langste dag van
  de zomer\n{]}}
\item Actions, verb phrases: \emph{\n{[}het verzamelen van liquide
  middelen\n{]}}
\item Quantities, measurements: \emph{\n{[}20 graden\n{]} , \n{[}100
  MB\n{]} , \n{[}ongeveer 10 euro\n{]}}

  However, not every NP with a quantity is excluded, because the NP may describe a specific object that is referred to again:

  \emph{'En wij kregen als speciale missie om [vijf miljoen Nederlandse
  guldens]\textsubscript{1} uit de kluizen van de Nederlandsche Bank in
  Middelburg via Duinkerken naar Londen te brengen.
  De koers waartegen [ze]\textsubscript{1} in Whitehall konden worden
  ingewisseld tegen Engelse ponden, was [\dots]. [Het geld]\textsubscript{1}
  zat in twee zwarte koffers, verdeeld over achthonderd linnen zakjes.}

\item Idioms: \emph{Wat is er aan \n{[}de hand\n{]}?}

    \emph{[Hij] zag [Esmée] bij [het hoofdeinde] in \n{[}gesprek\n{]} met [een familielid].} (\emph{gesprek} has no determiner, it is not a specific identifiable conversation that can be referred to again)

\item Material, substances, and other non-specific mass nouns:\\
    \emph{[het deksel van \n{[}blank hout\n{]}]}

\end{itemize*}



\section{Coreference links}

Only a single type of coreference is annotated, indicating that
mentions refer to the same entity. There is no annotation of the
specific antecedent for an anaphor; by linking mentions, they become
part of the same cluster and are considered equivalent. For example,
given a cluster \{John, he\} and a new mention ``him'', linking the new
mention to ``John'' or ``he'' makes no difference. Mentions that belong to the
same cluster are indicated with subscripts. The following kinds of
coreference are recognized:

\begin{itemize*}
\item Identity, strict coreference

    \emph{[Jan]\textsubscript{1} ziet [Marie]\textsubscript{2} .
    [Hij]\textsubscript{1} zwaait naar [haar]\textsubscript{2}j .}

\item Predicate nominals

  \emph{{[}Jan{]}\textsubscript{1} is {[}een schrijver{]}\textsubscript{1} .}

\item Relative clauses

    \emph{[De burgemeester]\textsubscript{1} {[}die{]}\textsubscript{1} de vergadering opende was behoorlijk nors.}

    \emph{[Het huis]\textsubscript{2} [waar]\textsubscript{2} ik ben geboren.}

\item Appositions. If the first part is a name, mark separately:

  \emph{{[}Hu Jintao{]}\textsubscript{1} , {[}de president van China{]}\textsubscript{1} , hield een
  toespraak voor de VN.}
  
  But a modifier followed by a name is a single mention:
  
  \emph{{[}zeilster Carolijn Brouwer{]}}
  
\item Type-token coreference:

  {[}The man{]}\textsubscript{1} who gave {[}{[}his{]}\textsubscript{1} paycheck{]}\textsubscript{2} to
  his wife was wiser than {[}the man{]}\textsubscript{3} who gave {[}it{]}\textsubscript{2} to
  [{[}his{]}\textsubscript{3} mistress]\textsubscript{4}.
  
  The mentions in cluster 2 are not identical, but are tokens of the
  same type.
  
\item Time-indexed coreference:

  \emph{{[}Bert Degraeve{]}\textsubscript{1} , tot voor kort {[}gedelegeerd
  bestuurder{]}\textsubscript{1} , gaat aan de slag als {[}chief financial and
  administration officer{]}\textsubscript{1} .}
  
  Cluster 1 contains mentions whose coreference is only valid at
  specific times, but we do not annotate this distinction.
\item Bound anaphora:

  \emph{{[}Iedere man{]}\textsubscript{1} steekt wel eens {[}zijn{]}\textsubscript{1} nek uit.}
  
\end{itemize*}

Special cases:
\begin{itemize*}
\item Always annotate the intended referent. In case of nicknames or jokes,
    you may have to distinguish mentions of the real referent, and nicknames or jokes that refer to someone else.

\item Metonymy:

    \emph{De VS heeft meerdere doelen gebombardeerd. Moskou heeft woedend
  gereageerd.}

  ``\emph{Moskou}'' refers here not to the city, but to the government of
  Russia. We annotate the intended referent, not the literal meaning.

\item Use/mention distinction:\footnote{%
    \url{https://en.wikipedia.org/wiki/Use\%E2\%80\%93mention_distinction}}

    \emph{[Jan]\textsubscript{1} is rijk, [hij]\textsubscript{1} heeft [een Ferarri].
    [Jan]\textsubscript{2} is [een gangbare naam]\textsubscript{2}.}

    The second instance of \emph{Jan} refers to the name/word itself,
    not the person. This is sometimes indicated with quotation marks.

    \emph{Maar verdomd, op [pagina vier] wordt [de aankomst in [de Hauptstadt]] gemeld van [een 'prominenter, unabhängiger politischer Publizist aus den Niederlanden']\textsubscript{1}.
    [Politischer Publizist]\textsubscript{2}!
    [Dat etiket]\textsubscript{2} zal [ik]\textsubscript{1} tijdens [dit bezoek] zeker niet meer kwijtraken.}

    The first mention refers to the protagonist,
    but the second mention refers to the label.

\end{itemize*}


Several more complex phenomena are excluded:
\begin{itemize*}
\item VP coreference:

  \emph{\n{[}Mijn fiets was gestolen\n{]} . \n{[}Dat\n{]} vond ik
  jammer .}
  
  \emph{\n{[}Heeft u ook een nieuwsbericht\n{]} , dan vernemen wij
  \n{[}dat\n{]} graag .}
  
  Note that in addition to not annotating a link, these are not mentions because they do not refer to objects or persons.

\item Part/whole, subset/superset relations (bridging relations):

  \emph{In de Raadsvergadering is het vertrouwen opgezegd in {[}het
  college{]}\textsubscript{1}. In een motie is gevraagd aan {[}alle
  wethouders{]}\textsubscript{2} hun ontslag in te dienen .}

  While the entities of \emph{het college} and \emph{alle wethouders} are related, they are distinct entities, and we do not annotate such a bridging relation between entities.
  
\item Modality/negation:

  \emph{{[}Een partij als de CD\&V{]} is nou niet echt {[}het toonbeeld
  van sociale betrokkenheid{]}}

\end{itemize*}


\section{Differences with other annotation schemes}
\subsection{Differences with the Corea annotation scheme}
Cf. \citet{bouma2007corea}

\begin{itemize*}
\item Only a single type of coreference relation is annotated,
  corresponding to the types IDENT, PRED, BOUND. The BRIDGE relation
  (part/whole, subset/superset relation) is not annotated.
\item Mentions belong to coreference clusters which are equivalence
  classes; the specific antecedent of an anaphor is not annotated. The
  type of entity, the head of a mention, and the type of coreference
  relation are not part of the annotation.
\item Mentions are manually corrected: all mentions that refer to an entity
  are annotated, non-referential spans are not included as mentions.
\item Relative pronouns are considered mentions and coreferent.

  Corea: \emph{{[}President Alejandro Toledo{]}\textsubscript{1} reisde dit weekend naar
  Seattle voor een gesprek met {[}Microsoft topman Bill Gates{]}\textsubscript{2} .
  {[}Gates, die al jaren bevriend is met {[}Toledo{]}\textsubscript{1} {]}\textsubscript{2} ,
  investeerde onlangs zo'n 550.000 Dollar in Peru .}
  
  These guidelines: \emph{{[}President Alejandro Toledo{]}\textsubscript{1} reisde dit
  weekend naar Seattle voor een gesprek met {[}Microsoft topman Bill
  Gates{]}\textsubscript{2} . {[}Gates{]}\textsubscript{2} , {[}die{]}\textsubscript{2} al jaren bevriend is
  met {[}Toledo{]}\textsubscript{1} , investeerde onlangs zo'n 550.000 Dollar in
  Peru.}
  
  Motivation: it can be difficult to identify the complete relative
  clause, due to discontinuity or long parenthetical remarks. Annotating
  the NP before the relative pronoun avoids a lot of difficult cases.
  Such cases are both difficult for annotators as well as for automatic parsers.
  For example:

    \begin{itemize*}
    \item Relative clauses can be discontinuous:
    
      \emph{Ik kan in elk geval getrouw {[}de indrukken{]}\textsubscript{1} weergeven
      {[}die{]}\textsubscript{1} deze feiten hebben achtergelaten .}
      
    \item Relative clause can be arbitrarily long:
    
      \emph{En dit was {[}de Perry{]}\textsubscript{1} {[}die{]}\textsubscript{1} vroeg op die ochtend
      in mei , voordat de zon te hoog stond om nog te kunnen spelen , op de
      beste tennisbaan in het beste door de recessie getroffen vakantieoord
      in Antigua stond , met de Russische Dima aan de ene kant van het net
      en Perry aan de andere .}
      
    \end{itemize*}
    
\item Obligatory reflexives are annotated:

    \emph{{[}Jan{]}\textsubscript{1} scheert
  {[}zich{]}\textsubscript{1}}
  
\end{itemize*}


\subsection{Differences with the Newsreader annotation scheme}
Cf. \citet{schoen2014newsreader}
\begin{itemize*}
\item Entities are not restricted to a set of predefined types
    (person, organization, location, product, \dots)
\item Relative pronouns, discontinuous NPs, and appositions are annotated
  differently.
\end{itemize*}

\bibliographystyle{apalike}
\bibliography{references}
\end{document}