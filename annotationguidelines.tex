\RequirePackage{filecontents}
\begin{filecontents}{references.bib}
@techreport{bouma2007corea,
  title={The {COREA}-project: {Manual} for the annotation of coreference in {Dutch} texts},
  author={Bouma, Gosse and Daelemans, Walter and Hendrickx, Iris and Hoste, V{\'e}ronique and Mineur, A},
  year={2007},
  institution={University of Groningen},
  note={\url{https://www.researchgate.net/publication/252395718}}
} % FIXME: permanent link?

@techreport{schoen2014newsreader,
  title={NewsReader Document-Level Annotation Guidelines - {Dutch}},
  author={Schoen, Anneleen and van Son, Chantal and van Erp, Marieke and van Vliet, Hennie},
  year={2014},
  institution={VU University},
  note={\url{http://www.newsreader-project.eu/files/2013/01/8-AnnotationGuidelinesDutch.pdf}}
}

@inproceedings{roesiger2018literary,
    title = {Towards Coreference for Literary Text: Analyzing Domain-Specific Phenomena},
    author = {R\"osiger, Ina  and Schulz, Sarah  and Reiter, Nils},
    year={2018},
    booktitle={Proceedings of LaTeCH-CLfL},
    pages={129--138},
    note = {\url{http://aclweb.org/anthology/W18-4515}}
}
\end{filecontents}
\PassOptionsToPackage{hyphens}{url}
\documentclass[a4paper]{article}
\usepackage[T1]{fontenc}
\usepackage[utf8]{inputenc}
\usepackage{kpfonts, mdwlist, microtype, xcolor, natbib}
\usepackage[unicode=true]{hyperref}
\hypersetup{pdfborder={0 0 0}, breaklinks=true}
\setcitestyle{round}
\setlength{\emergencystretch}{3em}  % prevent overfull lines
\newcommand{\n}[1]{\textcolor{red}{#1}}

\title{Annotation guidelines for dutchcoref}
\author{Andreas van Cranenburgh}
\date{March 27, 2019}  % \date{\today}

\begin{document}
\maketitle

\section{How to annotate?}

\begin{itemize*}
\item Read the text from start to finish, make and correct annotations as
  you go.
\item Identify mentions by asking yourself whether a span of text describes
  a specific identifiable object or person.
\item When the same entity is referred to again, ensure that both mentions
  are in the same coreference cluster. Conversely, remove any
  incorrect links.
\end{itemize*}

\section{Mentions}

A mention is a span of text that refers to an identifiable
entity or person in the real or mental world.
All mentions referring to objects or persons are
annotated, including entities that are not referred to again
(singletons). Mentions have been automatically identified, but they may
need to be corrected.

The correct span for mentions is indicated with square brackets {[} and {]};
a span that should not be annotated as mention is indicated with \n{[} red brackets\n{]}.

Due to technical limitations (i.e., the CoNLL 2012 format),
mentions must follow word boundaries. Therefore we must annotate:\footnote{%
    Fixing this would require correcting the tokenization and parse tree,
    which is outside the scope of these annotation guidelines.}

\emph{[hoek [Groot Hertoginnelaan-Laan] [van Meerdervoort]]} \\
and not: \emph{[hoek [Groot Hertoginnelaan\n{]}-\n{[}Laan van Meerdervoort]]}

\noindent The following subsections list the types of mentions that should be annotated.

\subsection{Pronouns}
\begin{itemize*}
    \item Personal pronouns (\emph{zij, hun}, \dots).
        Includes \emph{het} when used as pronoun.
    \item Possessive pronouns (\emph{mijn, zijn}, \dots)
    \item Demonstrative pronouns (\emph{die, dat, deze, dit, daar})
    \item Relative pronouns (\emph{die, dat, wie, wat})
    \item Reflexive/reciprocal pronouns (\emph{zich, zichzelf, elkaar}).
        Both obligatory and normal reflexives are annotated.
    \item Indefinite/generic pronouns (\emph{men, je, ze, iedereen, iemand, \dots})
      when the same unspecified person/object can be referred to again.
      This excludes e.g., negations \emph{niemand}
      or wh-pronouns in questions (\emph{wie, wat, welke, \dots)}.
    \item Pronominal adverbs of location: \emph{er, hier, daar, waar, waarin, \dots}
\end{itemize*}

Exclude non-referential, pleonastic, and/or expletive pronouns:
\begin{itemize*}
   \item \emph{\n{[}Het\n{]} regent.}
   \item \emph{[Daar] moeten we \n{[}het\n{]} over hebben.}
   \item \emph{\n{[}Er\n{]} zit \n{[}niets anders\n{]} op.}
\end{itemize*}

Indefinite pronouns require judgment to determine
whether they refer to an identifiable person/object
and whether they can be referred to again:
\begin{itemize*}
    \item The word \emph{iets} often occurs in a negative context,
        such as \emph{nauwelijks, zonder, alsof}, which indicates
        that there is no identifiable referent:
        \begin{itemize*}
            \item Van [die schaamte] is nauwelijks \n{[}iets\n{]} terug te vinden in [[zijn] romans].
            \item {[}Esmée{]} nam [ze] in [de hand] , staarde er langdurig naar , legde [ze] zonder \n{[}iets\n{]} te zeggen terug op [de kist] .
            \item \emph{[Oscar] bleef in [de verte] turen alsof [hij] \n{[}iets\n{]} verwachtte : [een plots inzicht] , [een verklaring] ?}
        \end{itemize*}
    \item Conversely, in the following sentence there is a concrete referent: \emph{"[Ik] zie [iets]," zegt [Jan]. [Een eiland] verschijnt aan de horizon.}
\end{itemize*}


\subsection{Proper nouns (named entities)}
\begin{itemize*}
    \item One-word names: \emph{[Jan], [Amerika]}.
    \item Multiword names form a single mention: \emph{[Jan de Vries], [de Verenigde Staten]}.
    \item Geographical: \emph{{[}Los Angeles, [California]{]}}
    \item Possessive: \emph{[[Jans] moeder]}
\end{itemize*}


\subsection{Noun phrases (NPs)}
Always annotate the longest, most specific
continuous span describing a mention. What to include:
\begin{itemize*}
    \item Determiners: \emph{[het huis]} \\
        A possessive pronoun is a determiner,
        and is also its own mention:\\
            \emph{[[mijn] fiets]}\\
        Quantifiers are also determiners: [iedere buitenlandse televisiezender]
    \item Adjectives, nouns: \emph{[een warme kop thee]}
    \item Prepositional phrases modifying the noun:
        \emph{[kandidaat voor [de coalitie]]}.
    \item Noun phrases within noun phrases. See previous example.
        Since \emph{kandidaat} and \emph{coalitie} describe different
        entities, they are both annotated. On the other hand,
        there is no need to mark \emph{kandidaat} twice:

        \emph{[\n{[}kandidaat\n{]} voor [de coalitie]]}.

\end{itemize*}

Special cases:
\begin{itemize*}
\item Conjunctions (\emph{Jan en Marie}). Include the whole conjunction as
  mention only when it functions as a unit in the text;
  e.g., when referred to again as a single group bij a plural pronoun ``\emph{ze}''.
  By default, only the individual conjuncts \emph{Jan} and \emph{Marie} are
  considered as separate mentions.

\item Disjunctions (\emph{tram 18 of 22}). Include the whole disjunction if there is a single intended referent; otherwise only annotate the disjuncts as mentions separately:

    \emph{[We] zijn in [Praag] op [de hoek van [de Vyšehradska] en [de Trojicka]]. [Tram 18 of 22] staat stil bij [de Botanische Tuin].}

    While the description is imprecise, the narrator has a specific tram in mind, so the whole is a mention; later, \emph{de tram} is used to refer to the same tram again.

    Contrast with the following, where there are two separate options,
    which do not form a single mention:

    \emph{Vanaf [het station] kan je [tram 18] of [22] nemen.}

\item NPs with commas.
    Except in special cases, a comma indicates the end of a mention:

    \emph{[De nieuwste iPhone]}, \emph{[een revolutionaire nieuwe smartphone]}.

    Special cases:
    \begin{itemize*}
        \item Geographical: \emph{{[}Los Angeles, [California]{]}}
        \item Adjectives: \emph{{[}Een mooie, rode roos{]}}
        \item Conjunction functions as group (see above)

          \emph{{[}{[}Jan{]}, {[}Marie{]} en {[}Joost{]}{]}}

    \end{itemize*}

\item Discontinuous NPs

  \emph{\n{[}[een belediging] /zijn/ van onze gastvrijheid\n{]}}

    Mentions must be continuous, uninterrupted spans in the text.
    Since the verb ``\emph{zijn}'' is not part of the noun phrase,
    it should also not be part of the mention.
    In this case only ``\emph{een belediging}'' is marked as a mention
    (i.e., the part with the head of the constituent \emph{belediging}).

\item Relative clauses and other NP-modifying clauses:
    The relative pronoun (\emph{die, dat, waar, waarover, \dots}) indicates the end of the mention:

    \begin{itemize*}
    \item \emph{\n{[}[De burgemeester]\textsubscript{1} {[}die{]}\textsubscript{1} de vergadering opende\n{]} was behoorlijk nors.}
    \item \emph{[Hij] en [ik], na [een aperitief]\textsubscript{1} [dat]\textsubscript{1} hij aanduidde als '[Rotkäppchen]\textsubscript{1}'}
    \item \emph{\dots [een kroeg op [de Schönhauser Allee]\textsubscript{1}] , [een buurt]\textsubscript{1} [waar]\textsubscript{1} volgens [hem] ondanks [de Wende] niets veranderd was .}
    \end{itemize*}

    The same holds for other clauses modifying an NP,
    e.g. \emph{[NP] om te \dots}:

    \emph{Dan overviel [mij] [de onweerstaanbare drang] om te vluchten , in [grote haast] , [de duivel] op [[mijn] hielen] . }

\end{itemize*}

What to exclude:
\begin{itemize*}
\item Time-related NPs: \emph{\n{[}gisteren\n{]} , \n{[}de langste dag van
  de zomer\n{]}}
\item Actions, manners, verb phrases: \emph{\n{[}het verzamelen van liquide
  middelen\n{]}}, \emph{\n{[}de wijze\n{]} \n{[}waarop\n{]} [die oude communisten] alles rechtpraten wat krom is}
\item Adjectives, demonyms: \emph{[de \n{[}Nederlandse\n{]} soldaten]}
\item Quantities, measurements: \emph{\n{[}20 graden\n{]} , \n{[}100
  MB\n{]} , \n{[}ongeveer 10 euro\n{]}}

  However, not every NP with a quantity is excluded, because the NP may describe a specific object that is referred to again:

  \emph{'En wij kregen als speciale missie om [vijf miljoen Nederlandse
  guldens]\textsubscript{1} uit de kluizen van de Nederlandsche Bank in
  Middelburg via Duinkerken naar Londen te brengen.
  De koers waartegen [ze]\textsubscript{1} in Whitehall konden worden
  ingewisseld tegen Engelse ponden, was [\dots]. [Het geld]\textsubscript{1}
  zat in twee zwarte koffers, verdeeld over achthonderd linnen zakjes.}

\item Idioms: \emph{Wat is er aan \n{[}de hand\n{]}?}

    \emph{[Hij] zag [Esmée] bij [het hoofdeinde] in \n{[}gesprek\n{]} met [een familielid].}
    (there is an implied conversation, but the common noun \emph{gesprek}
    is not a mention that can be referred to again)

\item Material, substances, and other non-specific mass nouns:\\
    \emph{[het deksel van \n{[}blank hout\n{]}]}

\item Descriptions in a negative context (\emph{niet, geen, nooit, \dots})
    do not refer, and are therefore not mentions:

    \emph{Maar nee, geen \n{[}glimmende regenjassen en gleufhoeden\n{]} 's nachts aan [de deur van [[mijn] hotelkamer]] , \n{[}geen enkele toespeling op [mijn] geschrijf\n{]} van de kant van [het Presseamt] , [waar] [ik] [mij] bij ieder bezoek aan [de DDR] nederig meldde , nooit \n{[}gezeur met visa\n{]} , \dots }
    %altijd even hoffelijk en hulpvaardig , [die naamloze , gezichtloze onderknuppels van [Herr Schulze]] , [die] natuurlijk ook niet zo heette . }
\end{itemize*}



\section{Coreference links}

Only a single type of coreference is annotated, indicating that
mentions refer to the same entity. There is no annotation of the
specific antecedent for an anaphor; by linking mentions, they become
part of the same cluster and are considered equivalent. For example,
given a cluster \{John, he\} and a new mention ``him'', linking the new
mention to ``John'' or ``he'' makes no difference. Mentions that belong to the
same cluster are indicated with subscripts. The following kinds of
coreference are recognized:

\begin{itemize*}
\item Identity, strict coreference

    \emph{[Jan]\textsubscript{1} ziet [Marie]\textsubscript{2} .
    [Hij]\textsubscript{1} zwaait naar [haar]\textsubscript{2}j .}


\item Predicate nominals

  \emph{{[}Jan{]}\textsubscript{1} is {[}een schrijver{]}\textsubscript{1} .}

\item Relative clauses

    \emph{[De burgemeester]\textsubscript{1} {[}die{]}\textsubscript{1} de vergadering opende was behoorlijk nors.}

    \emph{[Het huis]\textsubscript{2} [waar]\textsubscript{2} ik ben geboren.}

\item Appositions. If the first part is a name, mark separately:

  \emph{{[}Hu Jintao{]}\textsubscript{1} , {[}de president van China{]}\textsubscript{1} , hield een
  toespraak voor de VN.}

  But a modifier followed by a name is a single mention:

  \emph{{[}zeilster Carolijn Brouwer{]}}

\item Acronyms: [De Partij van de Arbeid]\textsubscript{1} ([PvdA]\textsubscript{1})

\item Generic entities. Only add a link when there is a clear anaphoric relation:

    \emph{[Men]\textsubscript{1} verloor [elkaar]\textsubscript{1} makkelijk uit het oog na [de Wende] .}

    If a later sentence mentions a generic \emph{men},
    annotate it as a different entity.

\item Type-token coreference:

  {[}The man{]}\textsubscript{1} who gave {[}{[}his{]}\textsubscript{1} paycheck{]}\textsubscript{2} to
  his wife was wiser than {[}the man{]}\textsubscript{3} who gave {[}it{]}\textsubscript{2} to
  [{[}his{]}\textsubscript{3} mistress]\textsubscript{4}.

  The referents of 2 are not identical, but are tokens of the same type.

\item Time-indexed coreference:

  \emph{{[}Bert Degraeve{]}\textsubscript{1} , tot voor kort {[}gedelegeerd
  bestuurder{]}\textsubscript{1} , gaat aan de slag als {[}chief financial and
  administration officer{]}\textsubscript{1} .}

  Cluster 1 contains mentions whose coreference is only valid at
  specific times, but we do not annotate this distinction.
\item Bound anaphora:

  \emph{{[}Iedere man{]}\textsubscript{1} steekt wel eens {[}zijn{]}\textsubscript{1} nek uit.}

\end{itemize*}

Special cases:
\begin{itemize*}
\item Always annotate the intended referent. In case of nicknames or jokes,
    you may have to distinguish mentions of the real referent, and nicknames or jokes that refer to someone else.

\item Metonymy:

    \emph{De VS heeft meerdere doelen gebombardeerd. Moskou heeft woedend
  gereageerd.}

  ``\emph{Moskou}'' refers here not to the city, but to the government of
  Russia. We annotate the intended referent, not the literal meaning.

    However, this only holds when the intended referent is strictly equivalent. The following cases are not coreferent:
    \begin{itemize*}
    \item \emph{[westerse critici] hadden [het boek]\textsubscript{1} ([zevenhonderd pagina's]\textsubscript{2}) onder hoongelach verguisd als '[hagiografie van een meeloper]\textsubscript{1}'} \\
    (\emph{zevenhonderd pagina's} refers only to a physical aspect of
    the book)
    \item \emph{[De gastheer]\textsubscript{1} begon met [een loflied op [Nederland]\textsubscript{2}] , [[zijn]\textsubscript{1} aangenaamste buitenlandse post]\textsubscript{3} , en ook [[zijn]\textsubscript{1} laatste]\textsubscript{3}.} (\emph{Nederland} and \emph{buitenlandse post}, a country and a job, are not equivalent)
    \item \emph{[Hij]\textsubscript{1} werd voorgesteld als '[mein Mitarbeiter Herr \dots]\textsubscript{1}'
([Naam]\textsubscript{2} niet verstaan.)} (person and name are not equivalent)
    \end{itemize*}

\item Use--mention distinction:\footnote{%
    Cf.\ \url{https://en.wikipedia.org/wiki/Use\%E2\%80\%93mention_distinction}\\
    NB: `mention' in this terminology is used in a different sense as
    in these guidelines.
    }

    \emph{[Jan]\textsubscript{1} is rijk, [hij]\textsubscript{1} heeft [een Ferarri].
    [Jan]\textsubscript{2} is [een gangbare naam]\textsubscript{2}.}

    The second instance of \emph{Jan} refers to the name/word itself,
    not the person. This is sometimes indicated with quotation marks.

    \emph{Maar verdomd, op [pagina vier] wordt [de aankomst in [de Hauptstadt]] gemeld van [een 'prominenter, unabhängiger politischer Publizist aus den Niederlanden']\textsubscript{1}.
    [Politischer Publizist]\textsubscript{2}!
    [Dat etiket]\textsubscript{2} zal [ik]\textsubscript{1} tijdens [dit bezoek] zeker niet meer kwijtraken.}

    The first mention refers to the protagonist,
    but the second mention refers to the label.

\end{itemize*}


Several more complex phenomena are excluded:
\begin{itemize*}
\item VP/clausal coreference:

  \emph{\n{[}Mijn fiets was gestolen\n{]} . \n{[}Dat\n{]} vond ik
  jammer .}

  \emph{\n{[}Heeft u ook een nieuwsbericht\n{]} , dan vernemen wij
  \n{[}dat\n{]} graag .}

  In addition to not annotating a link, these are not mentions because
  they do not refer to objects or persons.
  In the following example, the clause is not a mention (see use--mention distinction above), \emph{deze opmerking} is a mention (a mental object),
  but again there is no coreference link:

  \emph{\n{[}``Ik ben onschuldig,''\n{]} zei hij. Na [deze opmerking] bleef het stil.}

\item Part/whole, subset/superset relations (bridging relations):

  \emph{In de Raadsvergadering is het vertrouwen opgezegd in {[}het
  college{]}\textsubscript{1}. In een motie is gevraagd aan {[}alle
  wethouders{]}\textsubscript{2} hun ontslag in te dienen .}

  The entities of \emph{het college} and \emph{alle wethouders} are related
  but distinct entities, and we do not annotate such a bridging relation between entities.

\item Modality/negation:
  \emph{{[}Een partij als de CD\&V{]} is nou niet echt \n{[}het toonbeeld
  van sociale betrokkenheid\n{]}}

\end{itemize*}


\section{Comparison with related annotation schemes}
\subsection{Differences with the Corea annotation scheme for Dutch}
Cf. \citet{bouma2007corea}

\begin{itemize*}
\item Only a single type of coreference relation is annotated,
  corresponding to the types IDENT, PRED, BOUND. The BRIDGE relation
  (part/whole, subset/superset relation) is not annotated.
\item Mentions belong to coreference clusters which are equivalence
  classes; the specific antecedent of an anaphor is not annotated. The
  type of entity, the head of a mention, and the type of coreference
  relation are not part of the annotation.
\item Mentions are manually corrected: all mentions that refer to a person or object
  are annotated (including singletons), non-referential spans are not included as mentions.
\item Relative pronouns are considered mentions and coreferent.

  Corea: \emph{{[}President Alejandro Toledo{]}\textsubscript{1} reisde dit weekend naar
  Seattle voor een gesprek met {[}Microsoft topman Bill Gates{]}\textsubscript{2} .
  {[}Gates, die al jaren bevriend is met {[}Toledo{]}\textsubscript{1} {]}\textsubscript{2} ,
  investeerde onlangs zo'n 550.000 Dollar in Peru .}

  These guidelines: \emph{{[}President Alejandro Toledo{]}\textsubscript{1} reisde dit
  weekend naar Seattle voor een gesprek met {[}Microsoft topman Bill
  Gates{]}\textsubscript{2} . {[}Gates{]}\textsubscript{2} , {[}die{]}\textsubscript{2} al jaren bevriend is
  met {[}Toledo{]}\textsubscript{1} , investeerde onlangs zo'n 550.000 Dollar in
  Peru.}

  Motivation: it can be difficult to identify the complete relative
  clause, due to discontinuity or long parenthetical remarks. Annotating
  the NP before the relative pronoun avoids a lot of difficult cases.
  Such cases are both difficult for annotators as well as for automatic parsers.
  For example:

    \begin{itemize*}
    \item Relative clauses can be discontinuous:

      \emph{Ik kan in elk geval getrouw {[}de indrukken{]}\textsubscript{1} weergeven
      {[}die{]}\textsubscript{1} deze feiten hebben achtergelaten .}

    \item Relative clause can be arbitrarily long:

      \emph{En dit was {[}de Perry{]}\textsubscript{1} {[}die{]}\textsubscript{1} vroeg op die ochtend
      in mei , voordat de zon te hoog stond om nog te kunnen spelen , op de
      beste tennisbaan in het beste door de recessie getroffen vakantieoord
      in Antigua stond , met de Russische Dima aan de ene kant van het net
      en Perry aan de andere .}

    \end{itemize*}

\item Obligatory reflexives are annotated:

    \emph{{[}Jan{]}\textsubscript{1} scheert
  {[}zich{]}\textsubscript{1}}

\end{itemize*}


\subsection{Differences with the Dutch Newsreader annotation scheme}
Cf. \citet{schoen2014newsreader}
\begin{itemize*}
\item Entities are not restricted to a set of predefined types
    (person, organization, location, product, \dots)
\item Relative pronouns, discontinuous NPs, and appositions are annotated
  differently.
\end{itemize*}

\subsection{The proposed annotation scheme of R\"osiger et al.~(2018)}
\citet{roesiger2018literary} propose an annotation scheme for German literary texts
which provided the inspiration for these annotation guidelines.

Commonalities:
\begin{itemize*}
\item Mentions are manually corrected.
\item Coreference annotation of entity clusters
    instead of binary anaphora-antecedent links.
    No annotation of link type.
\item NP mentions in idiomatic expressions are excluded.
\item Bridging relations are excluded.
\end{itemize*}

\filbreak
Differences:
\begin{itemize*}
\item Singleton mentions are included.
\item Non-nominal antecedents (VPs, clauses) are not annotated.
\item Generic mentions/entities do not receive a special label.
\item Group mentions/entities do not receive a special label.\\
    No relations between entities are annotated.
\item Subtoken annotation is not allowed, to be compatible with the CoNLL 2012 format.
\item Discontinuous mentions are not allowed, for the same reason.
\end{itemize*}


\bibliographystyle{plainnat}
\bibliography{references}
\end{document}
